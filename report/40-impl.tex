\chapter{Технологический раздел}
\label{cha:impl}

\section{Выбор языка и среды программирования}

В качестве языка программирования был выбран язык Си. Для сборки модуля использовалась утилита make. В качестве среды программирования был выбран VSCode.

\section{Реализация алгоритма проверки необходимости сокрытия файла}

В листинге \ref{code:getdents} приведена реализация алгоритма проверки необходимости сокрытия файл.

\begin{lstlisting}[label=code:getdents,caption=Реализация алгоритма проверки необходимости сокрытия файл]
static asmlinkage int fh_sys_getdents64(const struct pt_regs *regs)
{
	struct linux_dirent64 __user *dirent = (struct linux_dirent64 *)regs->si;
	struct linux_dirent64 *previous_dir, *current_dir, *dirent_ker = NULL;
	unsigned long offset = 0;
	int ret = real_sys_getdents64(regs);
	
	dirent_ker = kzalloc(ret, GFP_KERNEL);
	
	if ((ret <= 0) || (dirent_ker == NULL))
	{
		return ret;
	}
	
	copy_from_user(dirent_ker, dirent, ret);
	
	while (offset < ret)
	{
		current_dir = (void *)dirent_ker + offset;
		
		if (check_fs_hidelist(current_dir->d_name))
		{
			if (current_dir == dirent_ker)
			{
				ret -= current_dir->d_reclen;
				memmove(current_dir, (void *)current_dir + current_dir->d_reclen, ret);
				continue;
			}
			
			previous_dir->d_reclen += current_dir->d_reclen;
		}
		else
		{
			previous_dir = current_dir;
		}
		
		offset += current_dir->d_reclen;
	}
	
	copy_to_user(dirent, dirent_ker, ret);
	
	kfree(dirent_ker);
	return ret;
}
\end{lstlisting}

\section{Реализация алгоритма проверки разрешения на удаление файла}

Реализация алгоритма проверки разрешения на удаление файла представлена в листинге \ref{code:unlink}.

\begin{lstlisting}[label=code:unlink,caption=Реализация алгоритма проверки разрешения на удаление файла]
static asmlinkage long fh_sys_unlink(struct pt_regs *regs)
{
	long ret=0;
	char *kernel_filename = get_filename((void*) regs->si);
	
	if (check_fs_blocklist(kernel_filename))
	{
		ret=0;
		kfree(kernel_filename);
		return ret;
	}
	
	kfree(kernel_filename);
	ret = real_sys_unlink(regs);
	
	return ret;
}
\end{lstlisting}

\section{Реализация алгоритма проверки разрешения на чтение из файла}

Реализация алгоритма проверки разрешения на чтение из файла представлена в листинге \ref{code:read}.

\begin{lstlisting}[label=code:read,caption=Реализация алгоритма проверки разрешения на чтение из файла]
static asmlinkage long fh_sys_read(struct pt_regs *regs)
{
	long ret = 0;
	struct task_struct *task = current;
	if (task->pid == target_pid && regs->si == target_fd)
	{
		return 0;
	}
	ret = real_sys_read(regs);
	return ret;
}
\end{lstlisting}

\section{Реализация алгоритма проверки разрешения на запись в файл}

Реализация алгоритма проверки разрешения на запись в файл представлена в листинге \ref{code:write}.

\begin{lstlisting}[label=code:write,caption=Реализация алгоритма проверки разрешения на запись в файл]
static asmlinkage long fh_sys_write(struct pt_regs *regs)
{
	long ret = 0;
	struct task_struct *task = current;
	if (task->pid == target_pid && regs->si == target_fd)
	{
		return 0;
	}
	ret = real_sys_write(regs);
	return ret;
}
\end{lstlisting}

\section{Инициализация полей структуры ftrace\_hook}

Инициализация полей структуры ftrace\_hook представлена в листинге~\ref{code:ftracehook2}.

\begin{lstlisting}[label=code:ftracehook2,caption=Инициализация полей структуры ftrace\_hook]
static struct ftrace_hook demo_hooks[] = {
	HOOK("sys_write", fh_sys_write, &real_sys_write),
	HOOK("sys_read", fh_sys_read, &real_sys_read),
	HOOK("sys_open", fh_sys_open, &real_sys_open),
	HOOK("sys_unlink", fh_sys_unlink, &real_sys_unlink),
	HOOK("sys_getdents64", fh_sys_getdents64, &real_sys_getdents64)
};
\end{lstlisting}

\section{Makefile}

В листинге \ref{code:makefile} представлен Makefile.

\begin{lstlisting}[label=code:makefile,caption=Makefile]
CONFIG_MODULE_SIG=n
PWD := $(shell pwd)
CC := gcc
KERNEL_PATH ?= /lib/modules/$(shell uname -r)/build
ccflags-y	+= -Wall -Wdeclaration-after-statement

obj-m += my_module.o
casperfs-objs := main.o hooked.o 

all:
make -C $(KERNEL_PATH) M=$(PWD) modules

clean:
make -C $(KERNEL_PATH) M=$(PWD) clean
\end{lstlisting}
